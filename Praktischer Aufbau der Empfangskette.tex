\subsection{Ausrichten der Antenne}

Um die Antenne auf den Satelliten ausrichten zu können, muss dessen Position bekannt und durch geeignetes Koordinatensystem beschrieben sein. Dafür werden sogenannte astronomische Koordinatensystem verwendet, wovon es mehrere verschiedene gibt. Diese unterscheiden sich dabei in ihrem Ursprung und in der Ermittlung der Koordinaten.\newline

Gängige Systeme sind dabei:
\begin{itemize}
    \item Horizontales Koordinatensystem: Beim horizontalen Koordinatensystem wird die Position eines Himmelskörpers, einen Satelliten in diesem Fall, im Bezug auf einen Beobachter auf der Erde beschrieben. Es werden dabei zwei Hauptkoordinate, den Höhenwinkel $\varepsilon$ (Elevation)  und der Kurs $\varphi$ (Azimut) , verwendet.\cite{Starwalk}\cite{Satellitenkommunikation}
    \item Äquatoriales Koordinatensystem: Anders als bei horizontalen Koordinatensystem wird die Position des Satelliten beim äquatorialen Koordinatensystem im Bezug auf den Himmelsäquator beschrieben. Die beiden verwendeten Hauptkoordinaten sind die Deklation Dec und Rektaszension RAC.\cite{Starwalk}
    \item Ekliptales Koordinatensystem: Im ekliptikalen Koordinatensystem wird die Position eines Satelliten im Bezug auf die Ekliptik angegeben. Bei der Ekliptik handelt es sich um die scheinbare Bahn oder Bewegung der Sonne, welche sich laufe eines Jahres am Himmel der Erde abzeichnet. Das Koordinatensystem verwendet dafür die beiden Hauptkoordinaten ekliptikale Länge und ekliptikale Breite.\cite{Starwalk}-> Quelle für Ekliptik finden
\end{itemize}
Zum Ausrichten von Antennen auf Satelliten kann am besten das horizontale Koordinatensystem verwendet werden, da es die Postion des Satelliten aus dem Punkt des Beobachters beschreibt.\newline
