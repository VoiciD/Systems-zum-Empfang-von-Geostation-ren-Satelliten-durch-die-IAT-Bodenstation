
\documentclass[a4paper,12pt]{article}

% Pakete für Zeichencodierung und Spracheinstellungen
\usepackage[utf8]{inputenc}    % UTF-8 Zeichenkodierung
\usepackage[T1]{fontenc}       % Korrekte Darstellung von Umlauten

\usepackage[ngerman]{babel}    % Deutsche Spracheinstellungen
\usepackage{amsmath}
% Pakete für Layout und Formatierung
\usepackage{graphicx}          % Einbindung von Bildern
\usepackage{geometry}          % Anpassung der Seitenränder
\usepackage{float}             % Exakte Positionierung von Abbildungen
\usepackage{fancyhdr}          % Anpassung von Kopf- und Fußzeilen
\usepackage{titlesec}          % Anpassung der Überschriftenformatierung
\usepackage{tocloft}           % Anpassung des Inhaltsverzeichnisses
\usepackage{cite}
\usepackage{gensymb}

% Paket für das Literaturverzeichnis
\usepackage[backend=biber, style=ieee]{biblatex} % IEEE-Zitierstil mit Biber als Backend
\addbibresource{Literaturverzeichnis.bib}       % Einbindung der Literaturdatenbank

% Paket zur Nutzung der letzten Seitenzahl
\usepackage{lastpage} % Ermöglicht „Seite X von Y“-Format

% Seitenlayout
\geometry{
  top=2cm,    % Abstand zum oberen Rand
  bottom=3cm, % Abstand zum unteren Rand
  left=2.5cm, % Abstand zum linken Rand
  right=2.5cm % Abstand zum rechten Rand
}

% Kein Absatzeinzug
\setlength\parindent{0pt}

% Kopf- und Fußzeilen
\setlength\headheight{26pt}  % Höhe der Kopfzeile
\setlength\headsep{35pt}     % Abstand zwischen Kopfzeile und Text
\pagestyle{fancy}
\fancyhf{}
\lhead{Daniel Albinger (5183249)}
\chead{Bachelor Thesis}
\rhead{\includegraphics[width=4cm]{Logo_HSB_Hochschule_Bremen.png}}
\cfoot{} % Keine Seitenzahl auf Titelseite und Inhaltsverzeichnis

\begin{document}

% Titelseite
\begin{titlepage}
    \centering
    \includegraphics[width=0.6\textwidth]{Logo_HSB_Hochschule_Bremen.png}\\[1cm]
    {\scshape\LARGE Hochschule Bremen\\}
    {\scshape\Large Fakultät 4: Elektrotechnik und Informatik\\[1.5cm]}
    {\huge\bfseries Bachelor Thesis\\[0.5cm]}
    {\vspace{20mm}}
    {\Large\bfseries Systems zum Empfang von Geostationären Satelliten durch die IAT Bodenstation\\[0.5cm]}
    {\Large\bfseries Daniel Albinger (5183249)\\[2cm]}
    {\Large\bfseries Prüfer\\[0.5cm]}
    \begin{tabular}{l}
    1. Prüfer: Prof. Dr. Peik \\ 
    2. Prüfer: Laboringeneur \\
    Gastbegleitung: M. Rauwolf
    \end{tabular}\\[2cm]
    {\Large\bfseries Abgabe: Datum}\\[2cm]
    \vfill
\end{titlepage}


\newpage

% Inhaltsverzeichnis ohne Seitenzahl
\pagenumbering{gobble}  % Keine Seitenzahl auf Inhaltsverzeichnis
\tableofcontents

\newpage  

% Seitenzahlen beginnen hier im Format „Seite X von Y“
\pagenumbering{arabic}  
\setcounter{page}{1}  
\cfoot{Seite \thepage\ von \pageref{LastPage}}

\section{Theorie und Grundlagen}
\subsection{Was ist ein Satellit}
Bei einem Satelliten handelt es sich im allgemeinen Verständnis um ein Objekt, welches sich in einer Umlaufbahn um einen Himmelskörper, wie z.B ein Planet, Mond, Stern oder ähnliches befindet. Dabei kann der Satellit natürlichem oder künstlichen Ursprung sein.\cite{Satellite_Technology}\newline
Im weiteren Verlauf handelt es sich bei einem Satelliten um ein künstliches Objekt, welches sich in einer Umlaufbahn um die Erde befindet.\newline
Die erste Idee für einen Satelliten in einer geostationären Umlaufbahn stammt aus dem Jahren 1945. In diesem Jahr veröffentlichte der Autor Arthur C. Clarke im Magazin "Wireless World" einen Artikel, in welchem er die Bedeutung des geostationären Orbits beschreibt. Die Zeit, die ein Satellit in einer geostationären Umlaufbahn benötigt, um einmal um die Umlaufbahn zu durchlaufen, entspricht der Zeit, die die Erde für ein Rotation um sich selbst benötigt. Für einen Beobachter auf der Erdoberfläche steht der Satellit immer über dem gleichen Punkt auf der Erde. Weiterhin beschreibt Clarke die Idee eines Kommunikationssatelliten im geostationären Orbit. Mit dem richtigen Equipment könnte ein solche Satellit eine Möglichkeit für interkontinentalen Datenaustausch bieten.\cite{Satellite_Technology}\newline
\begin{figure}[H]
    \centering
    \includegraphics[width=0.5\linewidth]{Bilder/Sputnik 1.jpeg}
    \caption{Das Bild zeigt Sputnik 1 vor dem Start. Sputnik 1 ist eine Aluminiumkugel mit einem Durchmesser von 58 cm und wiegt etwa 83 kg\cite{Bild_Sputnik}\cite{Sputnik_1}}
    \label{Sputnik 1}
\end{figure}
Der erste Satellit starte am 04. Oktober 1957 von der damaligen UdSSR. Der Satellit mit Namen "Спутник (Sputnik 1)", was so viel wie Begleiter oder Trabant bedeutet, umkreiste die Erde alle 98 min. Ausgerüstet war Sputnik 1 mit zwei Antennenpaaren und Telekommunikationsequipment, mit welchem er über $20.005$ MHz und $40.002$ MHz kurze Signale aussendete. Diese konnten auf der ganzen Welt empfangen werden. Nach etwa 92 Tage verglühte Sputnik 1 beim Wiedereintritt in die Atmosphäre. \cite{Sputnik_1}\newline
Der erste amerikanische Satellit starte am 31. Januar 1958 mit dem Namen "Explorer 1". Explorer 1 ist der erste Satellit gewesen, welcher wissenschaftliches Equipment ab Bord hatte.\cite{NASA_Explorer1}
\begin{figure}[H]
    \centering
    \includegraphics[width=0.5\linewidth]{Bilder/1958_january_explorer_01_team_0.jpg}
    \caption{Der erste amerikanische Satellit "Explorer 1" wird von den drei Leuten gehalten, welche für den Erfolg des Satelliten verantwortlich waren. Links befindet sich Dr. William H. Pickering, in der Mitte Dr. James A. van Allen und rechts Dr. Wernher von Braun \cite{NASA_Explorer1} }
    \label{Explorer 1}
\end{figure}
An Bord von Explorer 1 befanden sich wissenschaftliches Equipment, unter anderem auch ein Messgerät für kosmische Strahlung. Mit diesem Messgerät sollte die Strahlung in der Atmosphäre der Erde gemessen werden. Explorer 1 erbrachte den Nachweis für des Van-Allen Strahlungsgürtels. Der Satellit umrundete die Erde alle 114 min in einer kreisförmigen Umlaufbahn, wobei diese den Satelliten bis auf 354 km an die Erde ran und 2515 km entfernt brachte. Explorer 1 war 203 cm lang, hatte einen Durchmesser von 15.9 cm und wog 14 kg. Am 23. Mai 1958 machte die Explorer 1 ihre letzte Übertragung und verglühte am 31. März 1970 beim Wiedereintritt in der Atmosphäre.\cite{NASA_Explorer1}\newline
In der heutigen Zeit gibt es viele verschiedene Arten an Satelliten, welche sich in ihrem Verwendungszweck und ihrem damit verbunden Equipment und Umlaufbahn unterscheiden.\newline
Ein paar Beispiele wären dabei:
\begin{itemize}
    \item Erdbeobachtungssatelliten: Erdbeobachtungssatelliten werden, wie der Name bereits aussagt, zur Beobachtung und Analyse der Erdoberfläche und Atmosphäre eingesetzt. Ein Beispiel für solche Satelliten wären Wettersatelliten, welche in den verschiedensten Umlaufbahnen zu finden sind. Diese werden meistens mit verschiedensten Kameras, welche Bilder von der Oberfläche der Erde und Wolkenformation aufnehmen. Die Bilder bieten eine mögliche Grundlage für die Wettervorhersage. Beispiele für Wettersatelliten sind unter andrem die NOAA und GOES Reihe der Amerikaner, die METOP und METOSAT Reihe der Europäer und die METEOR und Electro-L Reihe der Russen.\cite{DWD}
    \item Kommunikations- und Rundfunksatelliten: Dieser Art der Satelliten stellen verschiedenste Service im Bereich der Telekommunikation und Rundfunk bereit. Sie werden unter anderem zur Übertragung von Fernsehsignalen, Telefonie und Internet verwendet. Auch diese Satelliten sind in den verschiedensten Umlaufbahnen zu finden. Ein paar Beispiele wären die Starlink Reihe von SpaceX, Iridium von gleichnamigen Unternehmen Iridum, Inmarsat vom gleichnamigen Unternehmen Inmarsat und Astra von SES S.A. ASTRA.\cite{Satellite_Technology}\cite{Satellitenkommunikation}\cite{N2YO_SATLIST}
    \item Navigationssatelliten: Navigationssatelliten werden zur genauen Positionsbestimmung verwendet. Die Positionsbestimmung basiert dabei auf der der Triangulierung und Einwegentfernungsmessung. Zur Bestimmung werden Signale von drei oder mehreren Satelliten empfangen. Die Signale enthalten neben den genauen Koordinaten des Satelliten auch den genauen Zeitpunkt, an welchem die Signale versendet werden. Grundlage für den genauen Zeitbestimmung bilden Atomuhren.\cite{Satellitenkommunikation}\newline
    Zusammen mit der Empfangszeit wird dann die Laufzeit der einzelnen Signale ermittelt, woraus die zurückgelegt Entfernung ermittelt werden kann. Der Standpunkt ist dann der Schnittpunkt der Kugeln, deren Mittelpunkte die Satelliten und deren Radius der zurückgelegte Weg ist.\cite{Satellitenkommunikation}\newline
    Die Genauigkeit ist dabei proportional zur Anzahl zur Triangulation verwendeten Satelliten. Beispiele für solche Sternenkonstellationen sind das GPS der Amerikaner, das GLONASS der Russen und das europäische Galileo.\cite{Satellitenkommunikation}
    \item Amateurfunksatelliten: Amateurfunksatelliten bilden eine besondere Untergruppe der Kommunikationssatelliten. Sie werden meistens von Universitäten, Vereinigungen von Amateurfunkern oder ähnlichen Vereinen geplant, entwickelt, gebaut und betrieben, wobei die enge Budgets und technologischen Innovationen bewundernswert sind.\cite{Satellitenkommunikation}
    Eine solche Vereinigung ist AMSAT, welche mehrere Ableger weltweit hat. In Deutschland gibt es die AMSAT-DL, welche sich aus Funkamateuren, Ingenieuren, Wissenschaftlern, Stundenten und Raumfahrtenthusiasten zusammensetzt. Seit über 50 Jahren plant, entwickelt, baut und betreibt AMSAT-DL verschiedenste Satelliten, welche von Funktamateuren frei verwendet werden dürfen.\cite{AMSAT-DL}\newline
    Der erste Amateurfunksatellit "OSCAR-I" (Orbital Satellite Carrying Amateur Radio) starte am 12. Dezember 1961, vier Jahre vor dem ersten kommerziellen Kommunikationssatelliten "Early Bird". Die ersten OSCAR-I,-II und -III Satelliten funktionierten nur wenige Tage. Erst OSCAR-VI von der deutschen AMSAT (AMSAT-DL) schaffte es 4,5 Jahren lang zu arbeiten. Es folgten  weitere deutsche OSCARS, weltweit insgesamt mehr als 50 Stück.\cite{Satellitenkommunikation}\newline
    Weitere Meilensteine von AMSAT-DL sind die sogenannten Phase 3 Satelliten. Die Entwicklung dieser Satelliten began in den 70er Jahren. Das Ziel der Phase 3 Satelliten ist eine Generation von Erdsatelliten in einer hochelliptischen Umlaufbahn zu erschaffen. Gegenüber der bisherigen Satelliten würden diese eine weltweiten Benutzerkreis erschließen. Von den bisher vier gestarteten Phase 3 Satelliten, mit der Bezeichnung P3-A bis P3-D, sind nur noch P3-B und P3-D im Orbit. \cite{AMSAT-DL}\cite{Satellitenkommunikation}\newline
    Ein weiterer Meilenstein ist der erste Phase 4 Satellit. Bei dem Satelliten handelt es sich um die katarischen Rundfunk- und Kommunikationssatelliten Es'Hail-2, welcher neben dem Equipment zur kommerziellen Nutzung auch zwei Amateurfunktransponder an Bord hat, welche die ersten im geostationären Orbit sind.\cite{AMSAT-DL}\newline
    Aktuell beschäftigt sich AMSAT-DL mit der Entwicklung eines weiter Phase 3 Satelliten (P3-E), welcher als Testsatellit für die Technologien von P5A dienen soll. Bei P5-A handelt es sich um den ersten Phase 5 Satelliten welcher zukünftig zum Mars fliegen soll.\cite{AMSAT-DL}
\end{itemize}
Ein Satellit wird meistens für einen Verwendungszweck geplant, entwickelt und gebaut. Je nach Verwendungszweck und geplanter Lebensdauer wird ein geeigneter Orbit ausgewählt.
\subsection{Umlaufbahnen für Satelliten}

Satelliten werden entsprechend ihrer Mission in eine geeignete Umlaufbahn platziert. Verschiedene Missionen können verschiedene Umlaufbahnen erfordern. Die Umlaufbahnen unterscheiden sich dabei in Form (Kreis oder Ellipse), Höhe und Inklination zum Äquator. \cite{Satellitenkommunikation}\newline
\begin{figure}[H]
    \centering
    \includegraphics[width=0.5\linewidth]{Bilder/Umlaufbahnen.png}
    \caption{Abbildung zeigt verschiedene Umlaufbahnen für Satelliten\cite{Satellitenkommunikation}}
    \label{Umlaufbahnen}
\end{figure}

In der Abbildung \ref{Umlaufbahnen} zeigt verschiedene Umlaufbahnen, welche für unterschiedliche Satelliten verwendet werden. Satelliten für Erdbeobachtung oder Wetteranalyse werden meistens in Umlaufbahnen möglichst nah an der Erdoberfläche platziert. So kann die Auflösung der aufgenommen Bilder erhöht werden.\cite{Satellitenkommunikation}\newline
Auf der anderen Seite werden Kommunikations- und Rundfunksatelliten in höhere Umlaufbahnen gebracht. Mit einer höheren Umlaufbahn kann die Sichtdauer auf den jeweiligen Satelliten und auch der Abdeckungsbereich des jeweilige Satelliten kann erhöht werden. In Folge dessen kann auch die Anzahl an notwendigen Satelliten, um zum Beispiel eine weltweite Abdeckung zu erreichen, auf ein Minimum reduziert werden.\cite{Satellitenkommunikation}\newline
\subsubsection*{Die allgemeine Kreisbahn}
Damit ein Satellit in seiner jeweiligen vorgesehen Umlaufbahn bleibt, müssen die auf den Satelliten wirkenden Kräfte im Gleichgewicht bleiben.
\subsection{Postion eines Satelliten}

\subsection{Es’ Hail-2}
Bei Es'Hail-2 handelt es sich um einen Kommunikationssatelliten, welcher von dem katarischen Unternehmen Es'hailSat betrieben wird.\newline
Gebaut wurde der Satellit von der Mitsubishi Electric Company (Melco). Es'Hail-2 basiert dabei auf der Melco DS-2000 Plattform. Der Satellit starte am 15.11.2018 an Bord einer Falcon 9 Rakete vom Cape Canaveral Space Center in einen geostationären Testorbit um 24\degree East. Nach der Testphase wurde Es'Hail-2 in seinen endgültigen Geostationären Orbit um 25.9\degree East verschoben. Die geplanten Lebenszeit von Es'Hail-2 beträgt 15 Jahre.\cite{EsHail2}\newline
An Bord von Es'Hail-2 befinden sich insgesamt 72 verschiedne Transponder für die L-, S-, X- Ku- und Ka-Bänder. Die Hauptaufgabe des Satelliten ist es, die Regionen Nord Afrika und den mittleren Osten mit TV- und Telekommunikationsdiensten zu versorgen. Die Nutzer sind dabei private Haushalte, Unternehmen oder Regierungsorganisationen.\newline
Neben den Transponder für die kommerzielle Nutzung des Satelliten befinden sich auch zwei Amateurfunk Transponder an Bord von Es'Hail-2. Bei diesen Transponder handelt es sich um die ersten P4-A Transponder im Geostationären Orbit. Sie sind einer Zusammenarbeit zwischen Es'hailSat, dem Qatar Amateur Radio Club (QARS) und AMSAT Deutschland (AMSAT-DL) als technische Leitung entstanden. Vorhanden sind ein Schmallband- (Narrowband) und ein Breitband- (Wideband) Transponder. Die Transponder tragen den Rufnamen QO-100 (Qatar Oscar 100).\cite{EsHail2}\newline
\begin{figure}[H]
    \centering
    \includegraphics[width=0.5\linewidth]{Bilder/EsHail-2 Coverage.png}
    \caption{Abdeckungsbereich der Amateurfunktranponder von Es'Hail-2\cite{CoverageEsHail2Amateur}}
    \label{CoverageEsHail2Amateur}
\end{figure}
Die Karte in Abbildung \ref{CoverageEsHail2Amateur} zeigt den Abdeckungsbereich der Amateurfunktransponder von Es'Hail-2. Abgedeckt wird runter bis 5° Elevation, in einigen Bereichen auch bis 0°. Die Abdeckung reicht damit von Brasilien, Teile Grönlands und der Antarktis über Afrika und Europa bis nach Thailand. Je nach Region wird für den Optimalen Empfang eine Schüssel von 60cm bis 90 cm empfohlen. Die Regionen mittig in der Abdeckung benötigen nur eine Schüssel mit geringen Durchmesser. In den Regionen am Rand oder in regenreichen Gegenden sind größere Schüsseln vom Vorteil.\cite{EsHail2}.



\subsubsection{Schmallband Transponder}
Für den Schmallband Transponder wird ein linearer Transponder verwendet. Ein normaler Repeater empfängt ein Signal auf einer festen Frequenz und sendet dieses auch wieder auf einer Frequenz. Ein linearer Transponder hingegen empfängt und sendet Signale über ein breites Spektrum und hält  dabei ein festes Offset zwischen den empfangenen und gesendeten Signalen ein.\cite{EsHail2}\cite{linearTransponder}\newline
Lineare Transponder werden häufig für Amateur- und Satellitenfunk verwendet. Ein linearer Transponder empfängt ein breites Spektrum an Frequenzen. Im falle von Es'Hail-2 liegt der Uplink im S-Band zwischen $2400.005$ MHz und $2400.490$ MHz. Der lineare Transponder hat somit eine Bandbreite von $500$ kHz.\cite{EsHail2}\cite{linearTransponder}
\begin{itemize}
    \item Mittenfrequenz: $f_{center}$ = $2400.175$ MHz
    \item Bandbreite: $BW_{Narrow}$ = $0.5$ MHz
    \item Polarisation: RHCP
\end{itemize}
Die empfangen Signale sendet der lineare Transponder dann wieder mit einem festen Offset auf einem anderen Frequenzband. Der Downlink von Es'Hail-2 liegt im X-Band zwischen $10489.500$ MHz
und $10490$ MHz.\cite{EsHail2}\cite{linearTransponder}
\begin{itemize}
    \item Mittenfrequenz: $f_{center}$ = $2400.175$ MHz
    \item Bandbreite: $BW_{Narrow}$ = $0.5$ MHz
    \item Polarisation: RHCP
\end{itemize}
Wird ein Signal auf $2400.100$ MHz vom Transponder empfangen, wird es vom Transponder auf $10489.650$ MHz wieder versendet. Das Offset wird vom Transponder beibehalten.\cite{linearTransponder}
Der Schmallband Transponder ist eine Reihe an verschiedenen Kommunikationsarten geeignet. Allgeimein wird dieser Transponder für Sprachübertragungen, Morse Code oder digitale Übertragungen mit geringer Leistung und Bandbreite verwendet. Wichtig ist, dass die dafür vorgsehenen Frequenzen verwendet werden.\cite{EsHail2}\newline
\begin{figure}[H]
    \centering
    \includegraphics[width=0.75\linewidth]{Bilder/AMSAT-QO-100-NB-Transponder-Bandplan-Graph.png}
    \caption{Bandplan des Schmallband Transponder\cite{EsHail2NarrowbandBandplan}}
    \label{NB-Bandplan}
\end{figure}
Die Abbildung \ref{NB-Bandplan} zeigt den Bandplan des Up- und Downlinks des Schmallband Transponder. Die drei rot gekennzeichnet Bereiche sind drei Beacons, welche den Frequenzbereich des Transponder eingrenzen. In diesem Bereich darf nicht gesendet werden. Die Beacons senden Daten mit 400 Bit/s mit verschiedensten Modulationen.\cite{EsHail2}\newline
Der Bereich von $2400.005$ MHz bis $2400.040$ MHz im Uplink oder $10489.505$ bis $10489.540$ MHz im Downlink ist ausschließlich für Übertragungen mit einer CW-Modulation vorgesehen. Der nächste Bereich von $2400.40$ MHz bis $2400.080$ MHz(Up) und $10489.540$ MHz bis $10489.580$ MHz (Down) ist für schmallbandige digitale Kommunikation reserviert. Die maximale Bandbreite beträgt hier $500$ Hz. Auch der nächste Bereich von $2400.080$ MHz bis $2400.150$ MHz (Up) und $10489.580$ MHz bis $10489.650$ MHz (Down) ist für digitale Kommunikation vorgesehen. Allerdings beträgt die maximale Bandbreite in diesem Bereich 2700 Hz. Die Bereiche von $2400.150$ MHz bis $2400.245$ MHz (Up), $10489.650$ MHz bis $10489.745$ MHz (Down) und $2400.255$ bis $2400.350$ MHz (Up), $10489.755$ MHz bis $10489.850$ MHz (Down) ist für Signale mit einer Einseitenband-AM reserviert. Auch hier beträgt die maximale Bandbreite 2700 Hz. Der letzte Bereich von $2400.350$ MHz bis $2400.490$ MHz (Up) und $10489.850$ MHz bis $10489.990$ MHz ist für die Kommunikation mit verschiedensten Modulationen und für spezielle Anlässe vorgesehen. In diesem Bereich findet sich auch ein Kanal für Broadcast und ein Kanal für Notfälle. Auch finden in diesem Bereich verschiedenste Wettbewerbe statt. \cite{EsHail2NarrowbandBandplan}\newline
Für den Schmalltransponder gibt es noch weitere Regelungen. Zu einem sollte auf eine Frequenz Modulation, in egal welchem Bereich, verzichtet werden. Frequenz modulierte Signale nehmen in Vergleich zu anderen Modulaktionsarten eine größere Bandbreite ein und benötigen auch eine größere Sendeleistung. Da sowohl die Bandbreite als auch die Leistung stark limitiert ist, sollte auf die Frequenz Modulation verzichtet werden. Erwünscht sind schmallbandige Modulationsverfahren wie CW, Einseitenband-AM oder PSK\cite{BarkerFM}\cite{EsHail2}\newline
Unterhalb von $2400$ MHz, gekennzeichnet im Downlink durch den unteren CW-Beacon, darf keine Übertragung stattfinden. Für Amateur Satellitenfunk ist rechtlich nur der Bereich von $2400$ MHz bis $2450$ MHz vorgesehen und zulässig. So wird eine Störung anderer Dienste verhindert\cite{EsHail2}\cite{FrequenzplanBundesnetzagentur}
Von AMSAT wird zum senden eine Full-Duplex Operation vorgeschrieben. Das bedeutet, dass der jeweilige Funker gleichzeitig senden und empfangen muss.\cite{EsHail2} Diese Vorschrift wird für den nächsten Punkt wichtig.\newline
Weiterhin sollte die Sendeleistung so klein wie nötig gehalten werden. AMSAT empfiehlt die Sendeleistung in der gleichen größen Ordnung wie die der Beacon zu halten, da es sonst zu ungewünschte Verzerrungen und Übersteuerungen kommen kann.  Signale mit zu großer Leistung werden von der LEILA-Sirene im Downlink gekennzeichnet. LEILA ist ein Akronym und steht für Leistungs Limit Anzeige. Wird ein Signal mit der LEILA-Sirene gekennzeichnet, muss der jeweilige Funker seine Leistung umgehend reduzieren.\cite{EsHail2}\newline
Das System für LEILA befindet sich in der Bodenstation bei Doha und in Bochum. Aufgrund der Verzögerung kann keine Bandsperre eingesetzt werden. Da aber zum senden eine Full-Duplex Verbindung vorgeschrieben ist, ist eine akustische Warneinrichtung ausreichend.\cite{EsHail2}

\subsection{Mischer}

Ein Mischer ist ein elektrisches Bauteil, welches unter anderem zur Frequenzumsetzung verwendet wird. Bei der Frequenzumsetzung werden elektrische Signale in höhere oder niedrigere Frequenzbänder umgesetzt. Das ist einer der Gründe, warum Mischer ein breite Anwendung in der Kommunikationstechnik findet.\newline



\subsubsection{Anwendung von Mischern}
\subsubsection{Rauschen und Verluste}









 







\section{Theoretische Betrachtung der Empfangskette}
0\subsection{Antenne}
Die Antenne ist mit der wichtigste Bestandsteil der Empfangskette an der Satellitenbodenstation. Erst mit einer geeigneten Antenne ist es mögliche die Signale vom Satelliten, welcher ebenfalls eine Antenne braucht um die Signale zu senden, zu empfangen. Die Antenne wandelt die leitungsgebundene Welle um und strahlt diese in den freien Raum ab oder empfängt die Wellen im freien Raum und gibt diese an die Leitung ab. Sie ist also das Verbindungsglied zwischen der leitungsgebundenen Welle und der Welle im freien Raum.\newline
Die IEEE definiert eine Antenne als ein passives, lineares und reziprokes Bauelement, welches Radiowellen abstrahlen, als auch empfangen kann\cite{IEEE145-1993}\cite{Balanis_2005}.\newline
Eine Antenne kann über viele verschiedene Parameter beschrieben werden. Diese Parameter helfen dabei eine geeignete Antenne für die jeweilige Anwendung zu finden.

\subsubsection*{Nah- und Fernfeld}
Die Umgebung einer Antennen kann in mehrere kleinere Bereiche unterteilt werden. Im unmittelbaren Umfeld der Antenne liegt das Nahfeld, welches auch Freßnel-Bereich genannt\cite{Radartutorial-Nahundfernfeld} genannt werden kann.









\subsubsection*{Nah- und Fernfeld}
Der Bereich um die Antenne kann in mehrere Bereiche aufgeteilt werden. Im mittelbaren Umfeld liegt das Nahfeld, auch Fresnel-Breich genannt\cite{Radartutorial-Nahundfernfeld}, der Antenne. Neben den abgestrahlten elektromagnetische Wellen wirken hier auch starke stationäre Felder, welche ebenfalls von der Antenne ausgehen. Beschreiben lassen sich die Felder durch die maxwellschen Gleichungen. Im Nahfeld wird die Berechnung der Felder aufgrund der hohen Ordnungen der Polynome erschwert\cite{Radartutorial-Nahundfernfeld}. Aus diesem Grund werden die Strahlungscharakteristiken einer Antenne im Fernfeld bestimmt. \cite{Balanis_2005}.\newline
Das Fernfeld, auch Fraunhofer-Bereich genannt, ist geometrisch deutlich größer als das Nahfeld. Es beginnt da, wo sich die elektromagnetischen Wellen frei im Raum ausbreiten können. Der Übergang zum Fernfeld kann Näherungsweise bestimmt werden. Für Antennen, welche in ihren geometrischen Abmessung kleiner als ihre Wellenlänge $\lambda$ sind, gilt\cite{Radartutorial-Nahundfernfeld}:
\begin{equation}
    r_{fern}=2\cdot\lambda
    \label{Nahfeld}
\end{equation}
Bei größeren Antennen, zum Beispiel Parabolantennen, gilt\cite{Radartutorial-Nahundfernfeld}:
\begin{equation}
    r_{fern}=\frac{2\cdot L^2}{\lambda}
    \label{Fernfeld}
\end{equation}
Dabei gibt die Variable L die geometrische Abmessung der Antenne an. Als sichere Faustformel kann ab einem Abstand $r>5\cdot\lambda$ vom Fernfeld ausgegangen werden.\newline
Im Fernfeld existieren nur die Felder der elektromagnetische Welle, was die Berechnung der Felder deutlich vereinfacht. Die elektrische und magnetische Komponente der EM-Welle befinden sich Phase zu einander und stehen orthogonal zur Ausbreitungsrichtung. Über das Verhältnis vom elektrischen und magnetischen Feld kann der Freiraumwiderstand $\eta_0$ bestimmt werden.
\begin{equation}
    \eta_0=\frac{\left|\vec{E}\right|}{\left|\vec{H}\right|}=\sqrt{\frac{\mu_0}{\varepsilon_0}}=\mu_0\sqrt{\frac{1}{\mu_0\cdot\varepsilon_0}}=377\Omega
    \label{GleichungFreimraumwiderstand}
\end{equation}
Bis zur Entfernung $r=\frac{L^2}{2\cdot \lambda}$ um die Antenne liegt die sogenannte Rayleigh-Zone. In diesem Bereich strahlt Antenne nicht nur Energie ab, sondern nimmt auch einen Teil der abgestrahlten Energie als Blindleistung wieder auf.\cite{Radartutorial-Nahundfernfeld}

\subsubsection*{Antennen-/Richtdiagramm}
Ein Antennen- oder Richtdiagramm stellt die Strahlungscharakteristik einer Antenne grafisch dar. Die Strahlungscharakteristik einer Antenne umfasst dabei die Strahlungsleistungsdichte, die Feldstärke, Intensität, Richtfaktor, Phasenlage und Polarisation.\cite{Balanis_2005} In den meisten Fällen wird im Antennendiagramm allerdings die Intensität der abgestrahlten Energie oder ihre Feldstärke in Abhängigkeit der Richtung dargestellt\cite{Radartutorial-Antennendiagramm}. Da Antennen reziproke Elemente sind gilt ein Antennendiagramm gleichermaßen für das Senden und auch für das Empfangen mit der jeweiligen Antenne. Im Sendefall gibt das Antennendiagramm die richtungsabhängige Ausstrahlung der Antenne an und im Empfangsfall die richtungsabhängige Empfangsempfindlichkeit.\cite{Radartutorial-Antennendiagramm}\newline
Auch besteht die Möglichkeit die Strahlungscharakteristik der Antenne mithilfe einer mathematische Funktion zu definieren.\cite{Balanis_2005}\newline
\begin{figure}[H]
    \centering
    \includegraphics[width=0.5\linewidth]{Bilder/Antennendiagramm.png}
    \caption{Ein Beispiel für ein horizontales Antennendiagramm im Polarkoordinatensystem\cite{Radartutorial-Antennendiagramm}}
    \label{Antennendiagrammbeispiel}
\end{figure}
Für das Antennendiagramm kann in unterschiedlichen Formen und in verschiednen Ebenen dargestellt werden. Ein Antennendiagramm kann im 2D-Raum entlang der horizontalen (Azimuth), als auch entlang der vertikalen Ebene (Elevation) der Antenne erstellt werden. Auch kann ein Antennendiagramm im 3D-Raum erstellt werden. Die Abbildung \ref{Antennendiagrammbeispiel} zeigt ein horizontales Antennendiagramm im polaren Koordinatensystem.\newline
Neben dem polaren Koordinatensystem kann auch das kartesische Koordinatensystem verwendet werden, jedoch kann im polaren Koordinatensystem die Richtwirkung der Antenne besser dargestellt werden.
\cite{Radartutorial-Antennendiagramm}.
\subsubsection*{Haupt- und Nebenkeulen}\label{Keulen}
Im Antennendiagramm in Abbildung \ref{Antennendiagrammbeispiel} lassen sich verschiedene Muster in der Strahlungscharakteristik der Antenne erkennen, welche auch Keulen genannt werden. Dabei werden die Keulen weiter in Haupt- und Nebenkeulen unterteilt. \newline
Bei der Hauptkeule handelt es sich um den Bereich einer Antenne, in dessen Richtung am meisten Energie abgestrahlt oder, im Empfangsfall, empfangen wird.\cite{Balanis_2005}
Bei einigen Antennen können auch mehrere Hauptkeulen vorhanden sein. Ein Beispiel dafür sind Loop- oder Dipolantennen, welche zwei Hauptkeule im Antennendiagramm aufweisen. Diese Hauptkeulen sind im 180\degree versetzt zu einander. Die Hauptkeulen stellen die bevorzugte Anwendungsrichtung einer Antenne dar, egal ob die Antenne im Sende- oder Empfangsbetrieb verwendet wird.\newline
Die Nebenkeulen handelt es sich um alle Keulen, welche nicht die Hauptkeule darstellen. Diese sind jedoch deutlich kleiner und sollte auch so klein wie möglich sein. Nebenkeule sind meistens unerwünscht, da sie Enegie in ungewollte Richtungen abstrahlen und so weniger Energie durch die Hauptkeule abgestrahlt wird oder da sie im Empfangsfall dafür sorgen, dass die Antenne aus eventuell unerwünschten Richtungen Signale aufnimmt und so den Empfang stören.\cite{Balanis_2005}. Die größten beiden größten Nebenkeulen werden auch Seitenkeulen genannt.\cite{Balanis_2005}.\newline
Der Abstand von der Hauptkeule zur größten Nebenkeule ist die Nebenkeulendämpfung. Je größer der Wert ist, desto kleiner sind die Nebenkeulen. Die Nebenkeulendämpfung ist ein wichtiger Parameter für Richtantennen, da damit die Richtschärfe ausgedrückt werden kann.\newline
Die Haupt- und Nebenkeulen bilden sich bei jeder Antenne, welche kein isotropischer Kugelstrahler ist.

\subsubsection*{Strahlbreite}
Im Zusammenhang mit dem Strahlungsmuster einer Antenne kann ein weiterer Parameter hergeleitet werden. Die Stahlbreite beschreibt den Öffnungswinkel der Hauptkeule. Gemessen wird die Strahlbreite an zwei identischen Punkten auf beiden Seiten des Maximums der Hauptkeule\cite{Balanis_2005}.\newline
Oft verwendet wird die 3dB-Strahlbreite, auch Half-Power Beamwidth genannt. Diese wird von der IEEE definiert als der Winkel zwischen den zwei Punkten an der Hauptkeule, wo die abgestrahlte Leistung nur noch die Hälfte des Maximums der Hauptkeule beträgt\cite{Balanis_2005}.\newline
Es gibt auch noch andere Strahlbreite wie die First Null Beamwith (FNBW), diese findet aber in der Praxis keine große Anwendung\cite{Balanis_2005}.\newline
Die Strahlbreite ist gerade für Richtantennen ein wichtiger Parameter, da die Strahlbreite ihr Auflösungsvermögen beschreibt. Mit einer kleineren Strahlbreite kann im Empfangsfall eine größere Winkelauflösung erreicht werden. Eine größere Winkelauflösung hilft einer Antenne dabei zwischen mehreren benachbarten Strahlungsquellen zu unterscheiden. Mit einem größeren Öffnungswinkel neigt die Antenne dazu benachbarte Quellen als eine wahrzunehmen. Das kann für zum Beispiel Radaranlagen wichtig sein\cite{Balanis_2005}. Allerdings wachsen mit geringere Strahlbreite auch die Nebenkeulen, was unerwünschte Effekte, wie in \ref{Keulen} beschrieben, führt \cite{Balanis_2005}.







\subsubsection*{Antennengewinn, Richtfaktor und Wirkungsgrad}
Ein weiterer nützlicher Parameter, welcher für die Beschreibung von Antennen verwendet werden kann, ist der Antennengewinn $G$. Der Antennengewinn ist eng mit dem Richtfaktor und dem Wirkungsrad der Antenne verbunden\cite{Balanis_2005}.\newline
Eine reale Antenne strahlt die eingespeiste Leistung $P_S$ nicht gleichmäßig in alle Richtungen ab. Eine reale Antenne weißt bevorzugte Richtungen $(\phi,\theta)$ auf, gekennzeichnet durch die Haupt- und Nebenkeulen im Antennendiagramm, in welche sie die Leistung abstrahlt oder aus welcher sie Leistung aufnimmt.\newline
Im Sendefall entspricht der Antennengewinn $G$ dem Verhältnis der abgestrahlten Strahlungsleistungsdichte $S(\phi,\theta)$ der Antenne zu der abgestrahlten Strahlungsleistungsdichte $S_{ref}(\phi,\theta)$ einer Referenzantenne bei gleicher eingespeisten Leistung $P_S$, Richtung $(\phi,\theta)$ und Entfernung $r$\cite{Balanis_2005}.
\begin{equation}
    G(\phi,\theta)=\frac{S(r,\phi,\theta)}{S_{ref}(r,\phi,\theta)}
    \label{Grunddefinition Antennengewinn}
\end{equation}
Die Entfernung r kürzt sich aus der Gleichung raus. Sie ist für den Antennengewinn nicht entscheidend.\newline
Da Antennen reziproke Elemente sind gilt die Gleichung \ref{Grunddefinition Antennengewinn} gleichermaßen für den Empfangsbetrieb. Die Definition ist allerdings leicht anders. Im Empfangsbetrieb entspricht der Antennengewinn $G$ dem Verhältnis der empfangenen Leistung $P_E(\phi,\theta)$ der jeweiligen Antenne zu der empfangenen Leistung $P_{Eref}(\phi,\theta)$ einer Referenzantenne bei gleicher Sendequelle mit fester Sendeleistung $P_S$ und Entfernung $r$ und gleichen Empfangswinkel $(\phi,\theta)$.\newline
\begin{equation}
    G(\phi,\theta)=\frac{P_E(\phi,\theta)}{P_{Eref}(\phi,\theta)}
    \label{Antennengewinn Empfangsfall}
\end{equation}
Als Referenzantenne in beiden Fällen eine beliebige Antenne gewählt werden. In den meisten Fällen wird als Referenzantenne der isotrope Kugelstrahler verwendet. Allerdings kann auch der einfache hertzsche Dipol verwendet werden\cite{Balanis_2005}.\newline
Der isotrope Kugelstrahler ist eine rein theoretische Antenne. Der isotrope Kugelstrahler strahlt die eingespeiste Leitung $P_S$ in alle Richtungen gleichmäßig aus und empfängt auch aus allen Richtungen die gleiche Leistung $P_E$. Aus diesem Grund eignet sich der isotrope Kugelstrahler besonders gut als Referenzantenne. Für die Strahlungsleistungsdichte eines isotrope Kugelstrahler gilt:
\begin{equation}
    S_0=\frac{P_S}{4\cdot \pi \cdot r^2 }
    \label{Strahlungsleistungsdichte isotroper Kugelstrahler}
\end{equation}
Der Gewinn wird meistens logarithmisch in dBi angegeben. Das i in dBi bedeutet, dass der Gewinn auf einen isotropen Kugelstrahler bezogen angeben wird. Aus der Gleichung \ref{Grunddefinition Antennengewinn} und \ref{Strahlungsleistungsdichte isotroper Kugelstrahler} folgt dann für die logarithmische Darstellung:
\begin{equation}
    G=10 \cdot \log_{10}\left( \frac{S(\phi,\theta)\cdot 4\cdot \pi \cdot r^2}{P_S} \right)
    \label{}
\end{equation}
















\subsubsection*{EIRP}
\subsubsection*{Polarisation}
\subsubsection*{Effektive Antennenfläche}
\subsubsection*{Antennenfaktor}
\subsubsection*{Antennengewinn}














 



\section{Praktischer Aufbau der Empfangskette}
\subsection{Ausrichten der Antenne}

Um die Antenne auf den Satelliten ausrichten zu können, muss dessen Position bekannt und durch geeignetes Koordinatensystem beschrieben sein. Dafür werden sogenannte astronomische Koordinatensystem verwendet, wovon es mehrere verschiedene gibt. Diese unterscheiden sich dabei in ihrem Ursprung und in der Ermittlung der Koordinaten.\newline

Gängige Systeme sind dabei:
\begin{itemize}
    \item Horizontales Koordinatensystem: Beim horizontalen Koordinatensystem wird die Position eines Himmelskörpers, einen Satelliten in diesem Fall, im Bezug auf einen Beobachter auf der Erde beschrieben. Es werden dabei zwei Hauptkoordinate, den Höhenwinkel $\varepsilon$ (Elevation)  und der Kurs $\varphi$ (Azimut) , verwendet.\cite{Starwalk}\cite{Satellitenkommunikation}
    \item Äquatoriales Koordinatensystem: Anders als bei horizontalen Koordinatensystem wird die Position des Satelliten beim äquatorialen Koordinatensystem im Bezug auf den Himmelsäquator beschrieben. Die beiden verwendeten Hauptkoordinaten sind die Deklation Dec und Rektaszension RAC.\cite{Starwalk}
    \item Ekliptales Koordinatensystem: Im ekliptikalen Koordinatensystem wird die Position eines Satelliten im Bezug auf die Ekliptik angegeben. Bei der Ekliptik handelt es sich um die scheinbare Bahn oder Bewegung der Sonne, welche sich laufe eines Jahres am Himmel der Erde abzeichnet. Das Koordinatensystem verwendet dafür die beiden Hauptkoordinaten ekliptikale Länge und ekliptikale Breite.\cite{Starwalk}-> Quelle für Ekliptik finden
\end{itemize}
Zum Ausrichten von Antennen auf Satelliten kann am besten das horizontale Koordinatensystem verwendet werden, da es die Postion des Satelliten aus dem Punkt des Beobachters beschreibt.\newline



\section{Anhang}
\subsection{Literaturverzeichnis}
\printbibliography
\subsection{Abbildungsverzeichnis}
\listoffigures
\subsection{Tabellenverzeichnis}
\listoftables
\end{document}